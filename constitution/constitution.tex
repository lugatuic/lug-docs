\documentclass[letter]{report}
\usepackage{atbegshi}
\usepackage{charter}
\usepackage{hyperref}
\usepackage[parfill]{parskip}
\pagestyle{empty}

\hypersetup{
	linktoc=all
}

\begin{document}

% approval notice
\AtBeginShipout{\AtBeginShipoutUpperLeft{
		\put(\dimexpr\paperwidth-1cm\relax,-1.5cm){\makebox[0pt][r]{\bf{\framebox{DOCUMENT
						REVIEWED AND APPROVED FOR 2025-2026}}}}
	}}

\title{Basic Rules for the Governance of the Linux User Group at the University
	of Illinois at Chicago}
\author{Linux User Group\\
	University of Illinois at Chicago\\
	Chicago, IL\\
	United States}
\date{\today}
\maketitle


\begin{abstract}
	This document defines the operative principles for the Linux
	User Group at the University of Illinois at Chicago (UIC).

	\vfill
	This work is licensed under the Creative Commons Attribution-ShareAlike
	4.0 International License. To view a copy of this license, visit
	\url{https://creativecommons.org/licenses/by-sa/4.0/}.
\end{abstract}


\tableofcontents


\chapter{NAME AND OBJECTIVE}
\section{Name of Organization}
This organization shall be known as the Linux Users Group or ``LUG" for short.

\section{Objective}
The objective of this organization shall be to promote the use, understanding,
and awareness of the Linux operating system; in addition to open source software
and related technologies.


\chapter{MEMBERSHIP}
\section{Eligibility}
It is the policy of the University not to engage in discrimination or harassment
against any person because of race, color, religion, sex, pregnancy, disability,
national origin, citizenship status, ancestry, age, order of protection status,
genetic information, marital status, sexual orientation, gender identity, arrest
record status,  unfavorable discharge from the military, or status as a
protected veteran and to comply with all federal and state nondiscrimination,
equal opportunity, and affirmative action laws, orders, and regulations.

\section{Standing}
A member in the organization can be in good or poor standing. By default members
are in good standing upon joining the organization.

A member in good standing has all the rights and privileges granted to that
member by this document. To remain in good standing a member must:
\begin{itemize}
	\item Members shall contribute to meetings and activities.
	\item Members shall uphold and complete all duties and responsibilities
	      assigned them as officers and members.
\end{itemize}
It is the responsibility of the Executive Committee to track and decide on the
standing of members.

Penalties for members in poor standing shall be drafted and
maintained by the Executive Committee.

\section{Tiers}
There are three membership types organized into two tiers. These tiers are
University and Alumni. The three membership types are University-Active and
University-Passive under the University tier and Alumni under the Alumni tier.

\subsection{University Tier}
Membership in both University type tiers is open to any current student of UIC.

There are two types of membership under the University tier: University-Active
and University-Passive.

For a member in good standing to switch between the two types of University Tier
membership the member must submit a request to the Traditional Officers at least
24 hours prior to the next scheduled Executive Committee vote so that there is
time to change any necessary records. If the request comes within 24 hours
before the next Executive Committee vote then it will not take effect until
after the vote.

\subsubsection{University-Active}
University-Active tier membership exists for those members who want the
privileges and responsibilities of being an active voting member of the Linux
User Group.

\subsubsection{University-Passive}
University-Passive membership exists to allow individuals to enjoy full benefits
of being a member of the Linux User Group but do not want the responsibility of
being a voting member involved in the administration of the club.

\section{Dues}
The Executive Committee shall have the right to levy fair annual dues for
membership.

All University tier members must pay these dues or risk falling into poor
standing or potential expulsion from the organization at the discretion of the
Executive Committee.

Alumni members shall be exempt from any dues.

\section{Handling of Funds}
The President and Treasurer are financially responsible. These officers will be
the only members allowed to access the COF account. The advisor may not be the
third officer. All monies collected on/off campus associated with this
organization, MUST be deposited into the organization’s COF account.

\section{Expulsion}
A member may be expelled from the Linux User Group for any or all of the
following reasons:
\begin{itemize}
	\item Stealing or the abuse of funds.
	\item Physical or verbal abuse of other members.
	\item Any gross misconduct as so deemed by the Executive Committee.
\end{itemize}
In order for a member to be expelled a majority vote must be obtained from the
Executive Committee. This vote will be held over the course of seven days to
give members ample time to consider their decision and vote. At the end of the
voting period votes will be tallied and the necessary action will be taken.


\chapter{EXECUTIVE COMMITTEE}
The Executive Committee shall be the primary governing body of the Linux User
Group. It will conduct, on behalf of the Linux User Group, all business it deems
necessary and consist of three sets: Traditional Officers, Non-Traditional
Officers, and The Vox.

\section{Voting and Quorum}
All voting matters shall be decided by a majority vote of a quorum of at least
2/3 of the Executive Committee, one of the present members voting must not be a
traditional officer, so long as such a member exists, with the exception of any
process which this document describes should be decided by some other process.

\section{Offices and Officers}
Officer positions in the Linux User Group may be created and destroyed at the
will of the Executive Committee by a standard vote as described in the
\hyperlink{section.3.1}{\textit{Voting and Quorum}} section with exception to
the three UIC required positions as described in the section
\hyperlink{section.3.3}{\textit{Traditional Officers}}.

All officers shall have a term of one calendar year starting on the day elected.

No officer, traditional or otherwise, has any more decision making power than
any other member of the Executive Committee. The purpose of officers in the
Linux User Group is to keep things organized, documented, and to help ensure
that the Linux User Group stays on track of its goals as decided by the
Executive Committee. They do not hold power akin to leaders in other
organizations unless granted that power by the Executive Committee. Because of
this power structure there is no term limit on any office in the organization.

\section{Traditional Officers}
There shall be three offices which must be filled each calendar year by
University tier members of the Linux User Group. These positions are required by
the university and so therefore must be filled. These positions are President,
Vice-President, Treasurer, and Secretary. Each position may be filled by one
person and one person only. This person must not have any other Traditional
Officer position within the Linux Users Group.

\subsection{President}
The president's duties are as follows:
\begin{itemize}
	\item Ensure that the organization upholds the principles laid down in
	      this document and in any further governing document created by
	      the Executive Committee.
	\item To be an easy point of contact for people inside and outside the
	      organization on matters related to the organization.
	\item Monitor and track progress of goals set for the Linux User Group
	      by the Executive Committee.
\end{itemize}

\subsection{Vice-President}
The Vice-President shares the same duties as the President and it should be
decided shortly after election how best to divide those duties between the two
officers.

\subsection{Treasurer}
The Treasurer's duties are as follows:
\begin{itemize}
	\item Ensure the creation of a budget by the Executive Committee,
	      shortly after elections, for the next year.
	\item Monitor the budget of the organization and assure it is spent
	      appropriately.
	\item Keep track of all money coming into and leaving the organization.
\end{itemize}

\subsection{Secretary}
The Secretary's duties are as follows:
\begin{itemize}
	\item Prepare formal communications with CS Department.
	\item Check status of research, project, and interest groups.
\end{itemize}

\section{Non-Traditional Officers}
Any University tier member or User Group may be appointed by the Executive
Committee as a non-traditional officer in order to oversee or administer some
important process for the Linux User Group.

\section{The Vox}
The Vox consists of all University-Active members of the Linux User Group who
are in good standing and who are not officers in the Linux User Group,
traditional or otherwise, as their position already guarantees them a vote. Each
member of the Linux User Group has the same say in decision making as anyone
else and to further that point this document explicitly creates The Vox as a
part of the Executive Committee. This freedom and power also means members have
a high responsibility to participate to the fullest amount they can in the
goings on of the Linux User Group.


\chapter{EXECUTIVE COMMITTEE DECISIONS}
Any proposal brought to, or decision made by the Executive Committee should be
drafted in the form of a document called a Request For Comments, or RFC for
short, as a nod to the IETF RFC standards approval process.

RFCs should be relatively short well thought out descriptions of some policy,
rule, or regulation proposed to the Executive Committee. This system exists to
ensure there is a well documented history of Executive Committee decisions for
reference on future actions.

All approved RFCs must be kept somewhere publicly accessible to all members of
the Linux User Group.

Any RFC which modifies, even temporarily, a policy that exists in another RFC
must contain a reference to that RFC so that one can easily cross reference
RFCs. The modified RFC must then be annotated with a reference to this change
and the new RFC. Any RFC found by the Executive Committee to violate this rule
is immediately invalidated until the RFC is corrected and re-voted on.

The Executive Committee shall hold a standard vote to decide on all proposed
RFCs.

This document may be used as a stylistic guide for the format of RFCs. All RFCs
should be given a unique identifying number incrementing from zero.


\chapter{USER GROUPS}
User groups are sub-units of the Linux User Group devoted to some business of
interest to one or more members of the Linux User Group. User groups can be
created for any purpose whether it be a general discussion group, a group in
charge of administering some business of the Linux User Group, or anything else
the members desire. Groups exist to have a framework for members of the Linux
User Group to self-organize and participate in and enrich the Linux User Group
community.

No member in good standing may be excluded from a group that that individual
wishes to join.

There are two tiers of Linux User Group: Unofficial and Official.

\section{Unofficial Groups}
Unofficial User Groups may be formed by any member(s) in good standing of the
Linux User Group at any time. Those wishing to create the group should gather
together and write a document defining what the group is, what it will do, and
how it will be run. This document should be iteratively improved over time with
the eventual goal of it becoming the official proposal for this unofficial group
to become an official group.

Unofficial groups may be granted funds by the Executive Committee upon request
but are not guaranteed an annual budget.

\section{Official Groups}
Official User Groups are official sub-groups of the Linux User Group. Any
University tier member in good standing may submit a proposal to the Executive
Committee to create an Official User Group. This proposal should come in the
form of an official RFC and describe what the group is, what it will do, and how
it will be run.

Official User Groups are to be allotted an annual budget by the Executive
Committee.

Official User Groups are expected to keep careful records of their finances,
membership, and activities. Representatives of each group will prepare a report
of the status of the group, which should include the information kept in the
required group records described above, each year to the executive committee.


\chapter{MEETINGS}
There are two meetings that the Linux User Group holds each year. These meetings
must happen in the spring semester and should occur within two weeks of the
previous year's meeting of that type. The Traditional Officers shall be
responsible for scheduling and running these meetings.

The meetings are the Election Meeting and the Budget Meeting.

At the Election Meeting the Executive Committee shall hold a vote for each of
the Traditional Officer positions as well as any Non-traditional Officer
Positions that were created and deemed to require an election to appoint.

At the Budget Meeting the Executive Committee will vote to approve the Linux
User Group's annual budget as drafted by the Treasurer with input from other
Executive Committee members. This meeting must take place after the Election
Meeting but before the end of the semester in which the Election Meeting took
place.


\chapter{ADVISOR}
The Linux User Group shall not enlist the assistance of an advisor on any and
all projects. The advisor is to have an overseer role over the organization as
to allow students to maintain autonomy over the Linux User Group. The only time
the advisor is to be utilized is in relation to finances of projects that would
require university funds and they (the current administration) are unsure as to
whether the Center of Student Involvement would be able to secure the funds on
student request alone. The Linux User Group advisor shall not oversee any
internal affairs such as impeachments or emergency elections. The Linux User
Group advisor shall be a full-time faculty member at UIC.


\chapter{AMENDMENTS AND REVISIONS}
This document may be altered by a vote of the Executive Committee. To pass any
alteration a 2/3 majority vote must be taken of 3/4 of the Executive Committee.

\end{document}

% vim: set tw=80:
